\documentclass{article}

\begin{document}

\section{Academic Misconduct}

All suspicion of academic misconduct will be reported to the OSSJA.

Below, I talk about my policies regarding cheating, i.e. what constitutes cheating, what happens if you do cheat, etc. Ask me for clarification if you are unclear on any aspect of these policies. I spend a lot of time trying to make these policies clear, which is why this section is large. I do this \textit{because I do not want students to \textbf{unknowingly} violate the policy}, and this is a very difficult process; if you ask questions, then that helps me know what to clarify about these policies. Each quarter, there is always at least one student who unknowingly violates the above policies, underscoring the importance of familiarizing yourself with what is and is not OK.

Note that we have software for detecting excessive similarities between code submitted by all of you as well as code that can be found online. Some of this software is relatively well-known, and some of it is not. (Some of it I wrote myself.) In other words, it's not smart to cheat in the one area of study in which all cheating-detection software is invented.

Different instructors have different academic misconduct policies; what constitutes cheating in one instructor's class may not be viewed as significant by another instructor. It is your responsibility to have \textbf{\textit{read and understood}} both the OSSJA's Code of Academic Conduct (\href{https://supportjudicialaffairs.sf.ucdavis.edu/code-academic-conduct}{here}) and the Department of Computer Science's Academic Misconduct Policy (\href{https://cs.ucdavis.edu/undergraduate/current-majors/policies}{here}). It has sometimes been the case that someone violates one of the above policies due to ignorance. The most common examples are sharing one's code with another student, copying code from Chegg (or even \textit{viewing} code on Chegg), and uploading documents from this class to CourseHero. Do note that ignorance does not save you from consequences for cheating. If you do not understand any aspect of the policy, it is your responsibility to seek clarification.

\subsection{If You Are Reported to the OSSJA}

Being reported to the OSSJA and being found guilty are not the same thing; you will get a chance to explain your side to the OSSJA, and they will let me know what you say, and we will proceed from there, through the OSSJA. If I do report you to the OSSJA, you are not to contact me about the matter, except if you wish to confess. The OSSJA will eventually contact you in order to schedule a meeting. If you have been reported to the OSSJA and your case has not been resolved by the time I submit final grades, then your final grade will temporarily be a Y, as per university policy.

\subsection{What Constitutes Cheating on Programming Assignments}

\textbf{\textit{You should regard your code as you would regard an essay that you are assigned to write}}; just as you would not show another student your essay or explain to them how exactly you made your rhetorical arguments, you should not share code -- or specific details about your code -- to other students. Below are ways to \textit{help ensure} you never get suspected of cheating (this is not an exhaustive list):

\begin{itemize}[itemsep=0mm, parsep=0pt]
\item \textit{\textbf{A good rule of thumb is that  you should understand every single line of code that you type}} (ignoring lines of code that I give you and tell you that you do not need to understand). \textit{\textbf{You should understand why any line that you type is in your program.}}
\item I recommend avoiding working on programming assignments with anyone in this class. I recommend that when it comes to seeking help on programming assignments, you stick to use of the class resources (e.g. office hours, email) or tutors. We do expect tutors to be competent about avoiding ``helping'' too much, but sometimes, tutors can overstep a bit. If a tutor is telling you exactly what code to type, then you should probably stop seeking their help. The same logic applies to assistance from a family member.
\item Do not copy any code from online sources. Note that if you look at code online, stop looking at it, and \textit{then} type your code, \textit{then you could still be reported for academic misconduct}. You should \textit{go out of your way} to avoid being suspected of cheating, not test the boundaries of what you can and cannot get away with.
\item \textbf{\textit{Do not show or share your code with anyone. Do not type code for anyone.}}
    \begin{itemize}[itemsep=0mm, parsep=0pt]
    \item Do not share your code with anyone outside of the course or who is auditing the course either. I once had a case in which student A and student B had nearly identical code because student A shared code with student C, who was not in the class, and then student C shared that code with student B, and student A and student B didn't even know each other!
    \item If you share your code with another student and they copy it without you knowing they were going to copy it, then you are still in violation, because you were not supposed to share your code with them in the first place.
    \item If your intent is to help your friend by sharing your code with them, then you are not actually helping them, even if you think you are. (And needless to say, you are still in violation.) If your friend cannot pass this course without cheating, then you're not doing them a favor by letting them go on to the next level, where they will probably need to cheat even more since they don't have the skills that they were supposed to get from the prerequisite courses.
    \end{itemize}
\item If you do help someone, only focus on high-level (i.e. verbal, language-agnostic) descriptions of what they should do. Do not tell anyone what lines of code to write. Ideally do not just tell them what approach to use; guide them to an approach that they think may work. \textbf{\textit{Even if you and someone else discuss how to approach a problem, if you both write your code separately, you (perhaps surprisingly) won't end up writing excessively similar code, especially if you follow rules established above such as not telling each other what code to write and not showing each other code.}}
\item If you help a friend debug their code (which is OK \textit{to an extent}), avoid touching their keyboard or telling them what code to type. You can definitely help them find the bug, but they need to \textit{understand} how to fix the bug, and they need to type the code that fixes the bug.
\end{itemize}

\subsection{Why You Should Not Cheat}

Below are a list of reasons for which you should not cheat.

\begin{itemize}[itemsep=0mm, parsep=0pt]
\item It is immoral, and you should strive to avoid doing immoral acts.
\item Prerequisites are very important in computer science courses. If you manage to cheat your way through this course, then you will find yourself struggling in the next courses. If you cheat your way through those courses too, then you will find yourself struggling in coding interviews.
\item There is a good chance you will have me as an instructor more than once during your undergraduate experience, and that might be awkward for you if you have previously been caught cheating in a class taught by me.
\item You will get caught. Moreover, according to university policy, I can give you an F in the course if you cheat (as determined through the OSSJA), regardless of the severity of the offense, even if your final grade would have otherwise been an A+.
\end{itemize}

\end{document}
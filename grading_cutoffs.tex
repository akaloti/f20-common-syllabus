\documentclass{article}

\begin{document}

\subsection{Grading Cutoffs}

As stated \href{https://registrar.ucdavis.edu/records/grades/letter}{here}, letter grades say something about your performance in the class:

\begin{tabular}{|l|l|}
\hline
Grade & Description \\ \hline
A     & ``Excellent''     \\ \hline
B     & ``Good''     \\ \hline
C     & ``Fair''     \\ \hline
D     & ``Barely passing''     \\ \hline
F     & ``Not passing''      \\ \hline
\end{tabular}

\vspace{1em}
The letter grade cutoffs in this class are the default ones:

\begin{tabular}{|l|l|}
\hline
Grade & Cutoff \\ \hline
A+    & 97     \\ \hline
A     & 93     \\ \hline
A-    & 90     \\ \hline
B+    & 87     \\ \hline
B     & 83     \\ \hline
B-    & 80     \\ \hline
C+    & 77     \\ \hline
C     & 73     \\ \hline
C-    & 70     \\ \hline
D+    & 67     \\ \hline
D     & 63     \\ \hline
D-    & 60     \\ \hline
F     & 0      \\ \hline
\end{tabular}

\vspace{1em}
I do not bump you up to the next grade if you are ``close enough''. For example, if your final percentage at the end of the quarter is 82.99, then you get a B-, not a B.

I do not curve\footnote{When I say ``curve'', I am referring to adjusting grades in order to impose a certain grade distribution. There are times that I adjust certain parts of the grading, but such adjustments are made without consideration of what the resulting grade distribution would be.} for many reasons, including that I believe that curving is immoral and incentivizes you to hope that your classmates perform poorly. It is possible that some assignments will have extra credit, but do not count on this.

\end{document}
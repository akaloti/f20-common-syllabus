\documentclass{article}

\begin{document}

\section{Means of Communication/Assistance}

You can use the below to ask for help on homework, ask conceptual questions, etc.

Compared to an introductory programming course such as ECS 32A, the amount of ``help'' that we are willing to give is less. In ECS 32A, your instructor may have been more okay with giving you approaches to problems or telling you exactly what is wrong with your code. In more advanced coding courses such as ECS 32B, ECS 34, and ECS 36A, you are obviously still encouraged to ask for help, but it is expected that you build an ability to come up with your own approaches (and, perhaps, ask for advice on whether those approaches are good) and that you can use a debugger or print statements to find bugs (and ask for help if you still cannot find your bugs after a decent amount of effort). It would be bad -- not just for your education but for your job prospects / performance on interviews -- if we gave you the answers to everything, so we won't do that. Consequently, when asking for help, you should be sure to say what you have \textit{tried}.

\end{document}